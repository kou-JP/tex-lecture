\section{\LaTeX の基本}
このセクションから\LaTeX の書き方を解説していきます.以下では可変部分を<\textit{string}>のように<>で斜体を挟んで表します.半角スペースは\verb*| |,全角スペースは⬚で表します.
\subsection{コマンドと環境}
\LaTeX のコマンドは\verb|\|<\textit{command}>のようなバックススラッシュに文字列がくっついた形をしています.コマンドにはコントロール・ワードとコントロール・シンボルの2種類があります.コントロール・ワードはバックススラッシュの後に文字列が続くもので,コントロール・シンボルはバックススラッシュの後に記号が1文字だけのものです.コントロール・ワードやコントロール・シンボルという名前を覚える必要はありません.バックススラッシュと記号1文字からなるコマンドは他のコマンドと仕様が少し異なるということを覚えれば十分です.コントロール・ワードは\verb*| |や\verb*|{}|のような記号の前までの文字列をコマンドとして認識します.コントロール・ワードの後の半角スペーススペースは無視されます.コントロール・ワードの後に半角スペーススペースを入力する場合は\verb*|\ |または\verb*|~|を用います\footnote{@で解説しますが前者は改行可能で後者は改行禁止という違いがあります.}.つまり\verb*|\TeXnician|はエラーになり,\verb*|\TeX nician|や\verb*|\TeX{}nician|は\TeX nicianと出力され,\verb*|\TeX\ nician|は\TeX\ nicianと出力されます.