\section{\TeX,\LaTeX とは?}
\subsection{\TeX とは?}
\TeX はD. E. Knuthによって開発された組版システムです.組版は文字や図を紙面に配置することを意味する印刷用語です.\TeX はWordやInDesignのようなWYSIWYG\footnote{What you see is what you getの略で完成形と編集画面の表示が等しいもの}なソフトとは異なり,テキスト形式で入力します.\TeX において使用する独自のプログラミング言語を\TeX 言語と呼びます.\TeX はEを少し下げて字間を詰めて書き,それが難しい場合はTeXのようにeを小文字にしてTとXを大文字にして書きます.\TeX は技術や芸術を意味するギリシャ語の\ruby{τέχνη}{テクネー}に由来するので最後のXを/x/と発音します./x/の音が存在しない言語においては近い音で代用されます.例えば,英語では/k/で代用することが多いようです.日本語では/ɸ/で代用して「テフ」と読むか,英語と同様に/k/で代用して「テック」と読むことが多いです.
\subsection{\LaTeX とは?}
\LaTeX はL. Lamportによって開発されたマクロパッケージです.マクロパッケージとは\TeX 言語を用いて組版するのに便利な機能を実装したもので\LaTeX の他にはplain \TeX やCon\TeX tなどがあります.素の\TeX のみで組版するのは非常に大変なので通常これらのマクロパッケージを用います.このノートでは\LaTeX のみを対象として解説するので,\LaTeX は\TeX に機能を様々な追加したもので他にも色々な種類があるとだけ理解していれば十分です.