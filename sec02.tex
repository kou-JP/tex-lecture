\section{\LaTeX を使う}
\subsection{\TeX ディストリビューション}
\TeX に様々な関するツールを集めたものを\TeX ディストリビューションと言います.最も有名な\TeX ディストリビューションは\TeX\ Liveです\footnote{他にはMiK\TeX やかつて使われていた日本語用のW32\TeX などがあります.}.arXivも\TeX\ Liveを使用しています.

\subsection{Webで\LaTeX を使う}
\LaTeX はインストールしなくてもwebで使うことができます.日本で有名なのはCloud LaTeX\footnote{\url{https://cloudlatex.io}}とOverleaf\footnote{\url{https://www.overleaf.com}}です.この2つはどちらも\TeX\ Liveを使用しています.Cloud LaTeXは日本企業が運営しており完全に日本語で使うことができます.Overleafは世界で最も有名なweb上の\LaTeX システムです.Overleafは有料機能ではあるものの複数人で編集する機能があり便利です.ただしOverleafはそのままでは(u)\pLaTeX が使えません.

\subsection{ローカルで\LaTeX を使う}
ローカルでは\TeX ディストリビューションとテキストエディタを組み合わせて\LaTeX を使います.\TeX\ Liveは年次ごとにバージョンが新しくなります.\TeX\ Liveのバージョンが異なりうまくタイプセットできない場合もあるので,そういった場合に対処できるように最新版だけでなく過去バージョンのインストール方法も説明します.どれか1つのバージョンをインストールする場合はどのバージョンが良いのかという疑問もあるかもしれません.その場合は最新版をインストールするのもいいですが,物理学者の場合はarXivが使用している\TeX\ Liveに合わせるというのもありだと思います.ここからはOS別に説明していきます.

\subsubsection*{Windows}
Windowsでは\TeX ディストリビューションは素直に\TeX\ Liveをインストールするのがお勧めです\footnote{MSYS2,Cygwinへインストールすることもできます.WSLへインストールしたい場合はLinuxの項目を参照してください.}.\TeX\ Liveのインストールはネットワークインストーラを用いる方法とISOイメージを用いる方法があります.ネットワークインストーラは最新版は\url{https://mirror.ctan.org/systems/texlive/tlnet/install-tl-windows.exe}から過去バージョンは\url{https://texlive.texjp.org/2023/tlnet/install-tl-windows.exe}(2023の部分を欲しい年に書き換えてください.)からダウンロードできます.